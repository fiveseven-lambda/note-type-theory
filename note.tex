\documentclass[a4paper]{ltjsarticle}
\usepackage{amsmath}
\newcommand\set{\;\mathrm{set}}
\begin{document}
まず,以下の4つの判断を考える.
\begin{itemize}
  \item $A\set$を「$A$は集合である」と読む.
  \item $A=B\set$を「$A$と$B$は等しい集合である」と読む.
    これは前提として$A\set$と$B\set$を含む.
  \item $a \in A$を「$a$は集合$A$の要素である」と読む.
    これは前提として$A\set$を含む.
  \item $a = b \in A$を「$a$と$b$は集合$A$の等しい要素である」と読む.
    これは前提として$A\set$と$a \in A$と$b \in A$を含む.
\end{itemize}
\end{document}
